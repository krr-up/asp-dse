\documentclass[11pt]{beamer}
\usetheme{Warsaw}
\usepackage[utf8]{inputenc}
\usepackage{amsmath}
\usepackage{amsthm}
\usepackage{amsfonts}
\usepackage{amssymb}
\usepackage{stmaryrd}
\usepackage{courier}
\usepackage{xcolor}
\usepackage{relsize}
\usepackage{tikz}
\usepackage{listings}

\newcommand{\Underscore}{\textscale{1}{\textunderscore}}
\author[Müller et al.]{Luise Müller and Kai Neubauer and Philipp Wanko}
\title[DSE with ASP: Progress and Outlook]{DSE with ASP: Progress and Outlook}
%\setbeamercovered{transparent} 
%\setbeamertemplate{navigation symbols}{} 
%\logo{} 
%\institute{} 
\date{} 
%\subject{} 

\begin{document}
% The language definitions below require the following packages:
%
%     \usepackage{courier}
%     \usepackage{xcolor}
%     \usepackage{relsize}
%
%     % The language definitions below require the following packages:
%
%     \usepackage{courier}
%     \usepackage{xcolor}
%     \usepackage{relsize}
%
%     % The language definitions below require the following packages:
%
%     \usepackage{courier}
%     \usepackage{xcolor}
%     \usepackage{relsize}
%
%     \input{listings}
%     \renewcommand{\Underscore}{\textscale{1}{\textunderscore}}
%
% Without extra packages I had to provide an alternative underscore definition
% to get decent looking underscores.
%
% As an alternative to courier, it is also possible to use the `lmodern`
% package.
%
%     \usepackage[T1]{fontenc}
%     \usepackage{lmodern}
%     \usepackage{xcolor}
%     \usepackage{relsize}
%
%     \input{listings}
%
% When the python or clingo language is given, tex code can be embedded
% escaping it in `#( ... #)`, which is treated like a comment in Python.
% Furthermore, for logic programs the character sequence is `%%` is not output.
% This can be used to add labels at the end of lines. For example, in a logic
% program one can write
%
%     H :- B.%%#( \label{lst:rule} #)
%
% to refer to the rule from the document without having to worry about changing
% line numbers when refactoring. The annotated program will still be runnable
% with clingo.

\providecommand{\Underscore}{\textunderscore}

\lstdefinelanguage{clingo}{%
  basicstyle=\ttfamily,%
  keywordstyle=[1]\bfseries,%
  keywordstyle=[2]\bfseries,%
  keywordstyle=[3]\bfseries,%
  showstringspaces=false,%
  literate={_}{\Underscore}1 {\%\%}{}0,%
  escapeinside={\#(}{\#)},%
  alsoletter={\#,\&},%
  keywords=[1]{not,from,import,def,if,else,elif,return,while,break,and,or,for,in,del,and,class,with,as,is,yield,async},%
  keywords=[2]{\#const,\#show,\#minimize,\#base,\#theory,\#count,\#external,\#program,\#script,\#end,\#heuristic,\#edge,\#project,\#show,\#sum},%
  keywords=[3]{&,&dom,&sum,&diff,&show},%
  morecomment=[l]{\#\ },%
  morecomment=[l]{\%\ },%
  morestring=[b]",%
  stringstyle={\itshape},%
  commentstyle={\color{darkgray}}%
}

\lstdefinelanguage{python}{%
  basicstyle=\ttfamily,%
  keywordstyle=[1]\bfseries,%
  showstringspaces=false,%
  literate={_}{\Underscore}{1},%
  escapeinside={\#(}{\#)},%
  alsoletter={\#,\&},%
  keywords=[1]{not,from,import,def,if,else,elif,return,while,break,and,or,for,in,del,and,class,with,as,is,yield,async},%
  morecomment=[l]{\#\ },%
  morestring=[b]",%
  stringstyle={\itshape},%
  commentstyle={\color{darkgray}}%
}


%     \renewcommand{\Underscore}{\textscale{1}{\textunderscore}}
%
% Without extra packages I had to provide an alternative underscore definition
% to get decent looking underscores.
%
% As an alternative to courier, it is also possible to use the `lmodern`
% package.
%
%     \usepackage[T1]{fontenc}
%     \usepackage{lmodern}
%     \usepackage{xcolor}
%     \usepackage{relsize}
%
%     % The language definitions below require the following packages:
%
%     \usepackage{courier}
%     \usepackage{xcolor}
%     \usepackage{relsize}
%
%     \input{listings}
%     \renewcommand{\Underscore}{\textscale{1}{\textunderscore}}
%
% Without extra packages I had to provide an alternative underscore definition
% to get decent looking underscores.
%
% As an alternative to courier, it is also possible to use the `lmodern`
% package.
%
%     \usepackage[T1]{fontenc}
%     \usepackage{lmodern}
%     \usepackage{xcolor}
%     \usepackage{relsize}
%
%     \input{listings}
%
% When the python or clingo language is given, tex code can be embedded
% escaping it in `#( ... #)`, which is treated like a comment in Python.
% Furthermore, for logic programs the character sequence is `%%` is not output.
% This can be used to add labels at the end of lines. For example, in a logic
% program one can write
%
%     H :- B.%%#( \label{lst:rule} #)
%
% to refer to the rule from the document without having to worry about changing
% line numbers when refactoring. The annotated program will still be runnable
% with clingo.

\providecommand{\Underscore}{\textunderscore}

\lstdefinelanguage{clingo}{%
  basicstyle=\ttfamily,%
  keywordstyle=[1]\bfseries,%
  keywordstyle=[2]\bfseries,%
  keywordstyle=[3]\bfseries,%
  showstringspaces=false,%
  literate={_}{\Underscore}1 {\%\%}{}0,%
  escapeinside={\#(}{\#)},%
  alsoletter={\#,\&},%
  keywords=[1]{not,from,import,def,if,else,elif,return,while,break,and,or,for,in,del,and,class,with,as,is,yield,async},%
  keywords=[2]{\#const,\#show,\#minimize,\#base,\#theory,\#count,\#external,\#program,\#script,\#end,\#heuristic,\#edge,\#project,\#show,\#sum},%
  keywords=[3]{&,&dom,&sum,&diff,&show},%
  morecomment=[l]{\#\ },%
  morecomment=[l]{\%\ },%
  morestring=[b]",%
  stringstyle={\itshape},%
  commentstyle={\color{darkgray}}%
}

\lstdefinelanguage{python}{%
  basicstyle=\ttfamily,%
  keywordstyle=[1]\bfseries,%
  showstringspaces=false,%
  literate={_}{\Underscore}{1},%
  escapeinside={\#(}{\#)},%
  alsoletter={\#,\&},%
  keywords=[1]{not,from,import,def,if,else,elif,return,while,break,and,or,for,in,del,and,class,with,as,is,yield,async},%
  morecomment=[l]{\#\ },%
  morestring=[b]",%
  stringstyle={\itshape},%
  commentstyle={\color{darkgray}}%
}


%
% When the python or clingo language is given, tex code can be embedded
% escaping it in `#( ... #)`, which is treated like a comment in Python.
% Furthermore, for logic programs the character sequence is `%%` is not output.
% This can be used to add labels at the end of lines. For example, in a logic
% program one can write
%
%     H :- B.%%#( \label{lst:rule} #)
%
% to refer to the rule from the document without having to worry about changing
% line numbers when refactoring. The annotated program will still be runnable
% with clingo.

\providecommand{\Underscore}{\textunderscore}

\lstdefinelanguage{clingo}{%
  basicstyle=\ttfamily,%
  keywordstyle=[1]\bfseries,%
  keywordstyle=[2]\bfseries,%
  keywordstyle=[3]\bfseries,%
  showstringspaces=false,%
  literate={_}{\Underscore}1 {\%\%}{}0,%
  escapeinside={\#(}{\#)},%
  alsoletter={\#,\&},%
  keywords=[1]{not,from,import,def,if,else,elif,return,while,break,and,or,for,in,del,and,class,with,as,is,yield,async},%
  keywords=[2]{\#const,\#show,\#minimize,\#base,\#theory,\#count,\#external,\#program,\#script,\#end,\#heuristic,\#edge,\#project,\#show,\#sum},%
  keywords=[3]{&,&dom,&sum,&diff,&show},%
  morecomment=[l]{\#\ },%
  morecomment=[l]{\%\ },%
  morestring=[b]",%
  stringstyle={\itshape},%
  commentstyle={\color{darkgray}}%
}

\lstdefinelanguage{python}{%
  basicstyle=\ttfamily,%
  keywordstyle=[1]\bfseries,%
  showstringspaces=false,%
  literate={_}{\Underscore}{1},%
  escapeinside={\#(}{\#)},%
  alsoletter={\#,\&},%
  keywords=[1]{not,from,import,def,if,else,elif,return,while,break,and,or,for,in,del,and,class,with,as,is,yield,async},%
  morecomment=[l]{\#\ },%
  morestring=[b]",%
  stringstyle={\itshape},%
  commentstyle={\color{darkgray}}%
}


%     \renewcommand{\Underscore}{\textscale{1}{\textunderscore}}
%
% Without extra packages I had to provide an alternative underscore definition
% to get decent looking underscores.
%
% As an alternative to courier, it is also possible to use the `lmodern`
% package.
%
%     \usepackage[T1]{fontenc}
%     \usepackage{lmodern}
%     \usepackage{xcolor}
%     \usepackage{relsize}
%
%     % The language definitions below require the following packages:
%
%     \usepackage{courier}
%     \usepackage{xcolor}
%     \usepackage{relsize}
%
%     % The language definitions below require the following packages:
%
%     \usepackage{courier}
%     \usepackage{xcolor}
%     \usepackage{relsize}
%
%     \input{listings}
%     \renewcommand{\Underscore}{\textscale{1}{\textunderscore}}
%
% Without extra packages I had to provide an alternative underscore definition
% to get decent looking underscores.
%
% As an alternative to courier, it is also possible to use the `lmodern`
% package.
%
%     \usepackage[T1]{fontenc}
%     \usepackage{lmodern}
%     \usepackage{xcolor}
%     \usepackage{relsize}
%
%     \input{listings}
%
% When the python or clingo language is given, tex code can be embedded
% escaping it in `#( ... #)`, which is treated like a comment in Python.
% Furthermore, for logic programs the character sequence is `%%` is not output.
% This can be used to add labels at the end of lines. For example, in a logic
% program one can write
%
%     H :- B.%%#( \label{lst:rule} #)
%
% to refer to the rule from the document without having to worry about changing
% line numbers when refactoring. The annotated program will still be runnable
% with clingo.

\providecommand{\Underscore}{\textunderscore}

\lstdefinelanguage{clingo}{%
  basicstyle=\ttfamily,%
  keywordstyle=[1]\bfseries,%
  keywordstyle=[2]\bfseries,%
  keywordstyle=[3]\bfseries,%
  showstringspaces=false,%
  literate={_}{\Underscore}1 {\%\%}{}0,%
  escapeinside={\#(}{\#)},%
  alsoletter={\#,\&},%
  keywords=[1]{not,from,import,def,if,else,elif,return,while,break,and,or,for,in,del,and,class,with,as,is,yield,async},%
  keywords=[2]{\#const,\#show,\#minimize,\#base,\#theory,\#count,\#external,\#program,\#script,\#end,\#heuristic,\#edge,\#project,\#show,\#sum},%
  keywords=[3]{&,&dom,&sum,&diff,&show},%
  morecomment=[l]{\#\ },%
  morecomment=[l]{\%\ },%
  morestring=[b]",%
  stringstyle={\itshape},%
  commentstyle={\color{darkgray}}%
}

\lstdefinelanguage{python}{%
  basicstyle=\ttfamily,%
  keywordstyle=[1]\bfseries,%
  showstringspaces=false,%
  literate={_}{\Underscore}{1},%
  escapeinside={\#(}{\#)},%
  alsoletter={\#,\&},%
  keywords=[1]{not,from,import,def,if,else,elif,return,while,break,and,or,for,in,del,and,class,with,as,is,yield,async},%
  morecomment=[l]{\#\ },%
  morestring=[b]",%
  stringstyle={\itshape},%
  commentstyle={\color{darkgray}}%
}


%     \renewcommand{\Underscore}{\textscale{1}{\textunderscore}}
%
% Without extra packages I had to provide an alternative underscore definition
% to get decent looking underscores.
%
% As an alternative to courier, it is also possible to use the `lmodern`
% package.
%
%     \usepackage[T1]{fontenc}
%     \usepackage{lmodern}
%     \usepackage{xcolor}
%     \usepackage{relsize}
%
%     % The language definitions below require the following packages:
%
%     \usepackage{courier}
%     \usepackage{xcolor}
%     \usepackage{relsize}
%
%     \input{listings}
%     \renewcommand{\Underscore}{\textscale{1}{\textunderscore}}
%
% Without extra packages I had to provide an alternative underscore definition
% to get decent looking underscores.
%
% As an alternative to courier, it is also possible to use the `lmodern`
% package.
%
%     \usepackage[T1]{fontenc}
%     \usepackage{lmodern}
%     \usepackage{xcolor}
%     \usepackage{relsize}
%
%     \input{listings}
%
% When the python or clingo language is given, tex code can be embedded
% escaping it in `#( ... #)`, which is treated like a comment in Python.
% Furthermore, for logic programs the character sequence is `%%` is not output.
% This can be used to add labels at the end of lines. For example, in a logic
% program one can write
%
%     H :- B.%%#( \label{lst:rule} #)
%
% to refer to the rule from the document without having to worry about changing
% line numbers when refactoring. The annotated program will still be runnable
% with clingo.

\providecommand{\Underscore}{\textunderscore}

\lstdefinelanguage{clingo}{%
  basicstyle=\ttfamily,%
  keywordstyle=[1]\bfseries,%
  keywordstyle=[2]\bfseries,%
  keywordstyle=[3]\bfseries,%
  showstringspaces=false,%
  literate={_}{\Underscore}1 {\%\%}{}0,%
  escapeinside={\#(}{\#)},%
  alsoletter={\#,\&},%
  keywords=[1]{not,from,import,def,if,else,elif,return,while,break,and,or,for,in,del,and,class,with,as,is,yield,async},%
  keywords=[2]{\#const,\#show,\#minimize,\#base,\#theory,\#count,\#external,\#program,\#script,\#end,\#heuristic,\#edge,\#project,\#show,\#sum},%
  keywords=[3]{&,&dom,&sum,&diff,&show},%
  morecomment=[l]{\#\ },%
  morecomment=[l]{\%\ },%
  morestring=[b]",%
  stringstyle={\itshape},%
  commentstyle={\color{darkgray}}%
}

\lstdefinelanguage{python}{%
  basicstyle=\ttfamily,%
  keywordstyle=[1]\bfseries,%
  showstringspaces=false,%
  literate={_}{\Underscore}{1},%
  escapeinside={\#(}{\#)},%
  alsoletter={\#,\&},%
  keywords=[1]{not,from,import,def,if,else,elif,return,while,break,and,or,for,in,del,and,class,with,as,is,yield,async},%
  morecomment=[l]{\#\ },%
  morestring=[b]",%
  stringstyle={\itshape},%
  commentstyle={\color{darkgray}}%
}


%
% When the python or clingo language is given, tex code can be embedded
% escaping it in `#( ... #)`, which is treated like a comment in Python.
% Furthermore, for logic programs the character sequence is `%%` is not output.
% This can be used to add labels at the end of lines. For example, in a logic
% program one can write
%
%     H :- B.%%#( \label{lst:rule} #)
%
% to refer to the rule from the document without having to worry about changing
% line numbers when refactoring. The annotated program will still be runnable
% with clingo.

\providecommand{\Underscore}{\textunderscore}

\lstdefinelanguage{clingo}{%
  basicstyle=\ttfamily,%
  keywordstyle=[1]\bfseries,%
  keywordstyle=[2]\bfseries,%
  keywordstyle=[3]\bfseries,%
  showstringspaces=false,%
  literate={_}{\Underscore}1 {\%\%}{}0,%
  escapeinside={\#(}{\#)},%
  alsoletter={\#,\&},%
  keywords=[1]{not,from,import,def,if,else,elif,return,while,break,and,or,for,in,del,and,class,with,as,is,yield,async},%
  keywords=[2]{\#const,\#show,\#minimize,\#base,\#theory,\#count,\#external,\#program,\#script,\#end,\#heuristic,\#edge,\#project,\#show,\#sum},%
  keywords=[3]{&,&dom,&sum,&diff,&show},%
  morecomment=[l]{\#\ },%
  morecomment=[l]{\%\ },%
  morestring=[b]",%
  stringstyle={\itshape},%
  commentstyle={\color{darkgray}}%
}

\lstdefinelanguage{python}{%
  basicstyle=\ttfamily,%
  keywordstyle=[1]\bfseries,%
  showstringspaces=false,%
  literate={_}{\Underscore}{1},%
  escapeinside={\#(}{\#)},%
  alsoletter={\#,\&},%
  keywords=[1]{not,from,import,def,if,else,elif,return,while,break,and,or,for,in,del,and,class,with,as,is,yield,async},%
  morecomment=[l]{\#\ },%
  morestring=[b]",%
  stringstyle={\itshape},%
  commentstyle={\color{darkgray}}%
}


%
% When the python or clingo language is given, tex code can be embedded
% escaping it in `#( ... #)`, which is treated like a comment in Python.
% Furthermore, for logic programs the character sequence is `%%` is not output.
% This can be used to add labels at the end of lines. For example, in a logic
% program one can write
%
%     H :- B.%%#( \label{lst:rule} #)
%
% to refer to the rule from the document without having to worry about changing
% line numbers when refactoring. The annotated program will still be runnable
% with clingo.

\providecommand{\Underscore}{\textunderscore}

\lstdefinelanguage{clingo}{%
  basicstyle=\ttfamily,%
  keywordstyle=[1]\bfseries,%
  keywordstyle=[2]\bfseries,%
  keywordstyle=[3]\bfseries,%
  showstringspaces=false,%
  literate={_}{\Underscore}1 {\%\%}{}0,%
  escapeinside={\#(}{\#)},%
  alsoletter={\#,\&},%
  keywords=[1]{not,from,import,def,if,else,elif,return,while,break,and,or,for,in,del,and,class,with,as,is,yield,async},%
  keywords=[2]{\#const,\#show,\#minimize,\#base,\#theory,\#count,\#external,\#program,\#script,\#end,\#heuristic,\#edge,\#project,\#show,\#sum},%
  keywords=[3]{&,&dom,&sum,&diff,&show},%
  morecomment=[l]{\#\ },%
  morecomment=[l]{\%\ },%
  morestring=[b]",%
  stringstyle={\itshape},%
  commentstyle={\color{darkgray}}%
}

\lstdefinelanguage{python}{%
  basicstyle=\ttfamily,%
  keywordstyle=[1]\bfseries,%
  showstringspaces=false,%
  literate={_}{\Underscore}{1},%
  escapeinside={\#(}{\#)},%
  alsoletter={\#,\&},%
  keywords=[1]{not,from,import,def,if,else,elif,return,while,break,and,or,for,in,del,and,class,with,as,is,yield,async},%
  morecomment=[l]{\#\ },%
  morestring=[b]",%
  stringstyle={\itshape},%
  commentstyle={\color{darkgray}}%
}


\newcommand{\sysfont}{\textit}

\newcommand{\Abstem}{\sysfont{Abstem}}
\newcommand{\aspcorei}{\sysfont{ASP-Core}}
\newcommand{\aspcoreii}{\sysfont{ASP-Core-2}}
\newcommand{\Clingo}{\sysfont{Clingo}}
\newcommand{\Gringo}{\sysfont{Gringo}}
\newcommand{\abstem}{\sysfont{abstem}}
\newcommand{\acthex}{\sysfont{acthex}}
\newcommand{\adsolver}{\sysfont{adsolver}}
\newcommand{\anthem}{\sysfont{anthem}}
\newcommand{\asparagus}{\sysfont{asparagus}}
\newcommand{\aspartame}{\sysfont{aspartame}}
\newcommand{\aspcud}{\sysfont{aspcud}}
\newcommand{\aspeed}{\sysfont{aspeed}}
\newcommand{\aspic}{\sysfont{aspic}}
\newcommand{\aspmt}{\sysfont{aspmt}}
\newcommand{\asprilo}{\sysfont{asprilo}}
\newcommand{\asprin}{\sysfont{asprin}}
\newcommand{\assat}{\sysfont{assat}}
\newcommand{\autofolio}{\sysfont{autofolio}}
\newcommand{\berkmin}{\sysfont{berkmin}}
\newcommand{\caspo}{\sysfont{caspo}}
\newcommand{\caspots}{\sysfont{caspo-ts}}
\newcommand{\chasp}{\sysfont{chasp}}
\newcommand{\chuffed}{\sysfont{chuffed}}
\newcommand{\claspD}{\sysfont{claspD}}
\newcommand{\claspar}{\sysfont{claspar}}
\newcommand{\claspfolio}{\sysfont{claspfolio}}
\newcommand{\claspre}{\sysfont{claspre}}
\newcommand{\clasp}{\sysfont{clasp}}
\newcommand{\clingcon}{\sysfont{clingcon}}
\newcommand{\clingo}{\sysfont{clingo}}
\newcommand{\clingodl}{\clingoM{dl}}
\newcommand{\clingolp}{\clingoM{lp}}
\newcommand{\cmodels}{\sysfont{cmodels}}
\newcommand{\coala}{\sysfont{coala}}
\newcommand{\cplex}{\sysfont{cplex}}
\newcommand{\dflat}{\sysfont{dflat}}
\newcommand{\dingo}{\sysfont{dingo}}
\newcommand{\dlvhex}{\sysfont{dlvhex}}
\newcommand{\dlv}{\sysfont{dlv}}
\newcommand{\eclingo}{\sysfont{eclingo}}
\newcommand{\embasp}{\sysfont{embasp}}
\newcommand{\ezcsp}{\sysfont{ezcsp}}
\newcommand{\ezsmt}{\sysfont{ezsmt}}
\newcommand{\fastdownward}{\sysfont{fastdownward}}
\newcommand{\ftolp}{\sysfont{f2lp}}
\newcommand{\gasp}{\sysfont{gasp}}
\newcommand{\gecode}{\sysfont{gecode}}
\newcommand{\gfd}{\sysfont{g12fd}}
\newcommand{\gidl}{\sysfont{gidl}}
\newcommand{\ginkgo}{\sysfont{ginkgo}}
\newcommand{\gnt}{\sysfont{gnt}}
\newcommand{\gringo}{\sysfont{gringo}}
\newcommand{\harvey}{\sysfont{harvey}}
\newcommand{\iclingo}{\sysfont{iclingo}}
\newcommand{\idlv}{\sysfont{idlv}}
\newcommand{\idp}{\sysfont{idp}}
\newcommand{\inca}{\sysfont{inca}}
\newcommand{\jdlv}{\sysfont{jdlv}}
\newcommand{\lctocasp}{\sysfont{lc2casp}}
\newcommand{\lparse}{\sysfont{lparse}}
\newcommand{\lpconvert}{\sysfont{lpconvert}}
\newcommand{\lpsolve}{\sysfont{lpsolve}}
\newcommand{\lptodiff}{\sysfont{lp2diff}}
\newcommand{\lptosat}{\sysfont{lp2sat}}
\newcommand{\mchaff}{\sysfont{mchaff}}
\newcommand{\measp}{\sysfont{measp}}
\newcommand{\metasp}{\sysfont{metasp}}
\newcommand{\mingo}{\sysfont{mingo}}
\newcommand{\minisatid}{\sysfont{minisatid}}
\newcommand{\minisat}{\sysfont{minisat}}
\newcommand{\minizinc}{\sysfont{minizinc}}
\newcommand{\mznfzn}{\sysfont{mzn2fzn}}
\newcommand{\nomorepp}{\sysfont{nomore++}}
\newcommand{\oclingo}{\sysfont{oclingo}}
\newcommand{\omiga}{\sysfont{omiga}}
\newcommand{\picatsat}{\sysfont{picatsat}}
\newcommand{\picat}{\sysfont{picat}}
\newcommand{\piclasp}{\sysfont{piclasp}}
\newcommand{\picosat}{\sysfont{picosat}}
\newcommand{\plasp}{\sysfont{plasp}}
\newcommand{\qasp}{\sysfont{qasp}}
\newcommand{\quontroller}{\sysfont{quontroller}}
\newcommand{\reify}{\sysfont{reify}}
\newcommand{\rosoclingo}{\sysfont{rosoclingo}}
\newcommand{\sag}{\sysfont{sag}}
\newcommand{\satz}{\sysfont{satz}}
\newcommand{\siege}{\sysfont{siege}}
\newcommand{\smac}{\sysfont{smac}}
\newcommand{\smodelscc}{\sysfont{smodels$_{\!cc}$}}
\newcommand{\smodelsr}{\sysfont{smodels}$_r$}
\newcommand{\smodels}{\sysfont{smodels}}
\newcommand{\stelp}{\sysfont{stelp}}
\newcommand{\sugar}{\sysfont{sugar}}
\newcommand{\teaspoon}{\sysfont{teaspoon}}
\newcommand{\tefoli}{\sysfont{tefoli}}
\newcommand{\telingo}{\sysfont{telingo}}
\newcommand{\tel}{\sysfont{tel}}
\newcommand{\unclasp}{\sysfont{unclasp}}
\newcommand{\wasp}{\sysfont{wasp}}
\newcommand{\xorro}{\sysfont{xorro}}
\newcommand{\zchaff}{\sysfont{zchaff}}
\newcommand{\zzz}{\sysfont{z3}}

\newcommand{\clingoM}[1]{\clingo{\small\textnormal{[}\textsc{#1}\textnormal{]}}}
\newcommand{\ASPm}[1]{ASP\raisebox{.7pt}{[\textsc{#1}]}}

\newcommand{\flatzinc}{\sysfont{FlatZinc}}
\newcommand{\aspif}{\sysfont{aspif}}

\newcommand{\C}{C}
\newcommand{\cpp}{C++}
\newcommand{\haskell}{Haskell}
\newcommand{\java}{Java}
\newcommand{\lua}{Lua}
\newcommand{\prolog}{Prolog}
\newcommand{\python}{Python}
\newcommand{\rust}{Rust}


\begin{frame}
\titlepage
\end{frame}

\section{Overview}

\begin{frame}{Overview}
\begin{itemize}
  \item Consolidated old work
  \item Reimplemented system with \clingo's application class
  \pause
  \item Currently working on experiments and writeup for \emph{Evolutionary System Design}
  \pause
  \item Collecting material and ideas for \emph{Generative Design Space Exploration}
\end{itemize}
\end{frame}

\section{Evolutionary System Design}

\begin{frame}{Problem Description}
  \begin{itemize}
    \item Given is an implementation of a specification
    \item Specification is modified
    \item Given new specification and legacy implementation, find implementations that are, first, Pareto-optimal and, second, close to the legacy implementation
  \end{itemize}
\end{frame}

\begin{frame}{Experimental Setup}
  \begin{enumerate}
    \item Find best-possible implementations to our benchmark set with a high timeout - currently running
    \item Select among the non-dominated implementations a random implementation - TODO
    \item Modify instance benchmark set to varying degrees - TODO
    \item Find implementations to new benchmarks with small timeout - TODO
    \item Employ strategies, optimization, domain-specific heuristic to find implementations to new benchmarks that are similar to legacy solution with same timeout as above - TODO
  \end{enumerate}
  \pause
  Expectation: We have faster convergence and smaller distance with similarity information, giving us high-quality solutions faster while being easier to produce.
\end{frame}

\begin{frame}{Strategies}
\end{frame}

\begin{frame}{Optimization}
\end{frame}

\begin{frame}{Heuristics}
\end{frame}

\section{Generative Design Space Exploration}

\begin{frame}{Problem Description}
  \begin{itemize}
    \item Architecture unknown prior to design process
    \item Instead of structure, we only have a set of hardware types (i.e., \textit{device library})
    \item As before, classified into processing and communication elements
    \item Mapping options assign tasks to processing \textit{types}, not to specific instances
    \item Topology of architecture is subject to exploration 
    \item Research questions:
    \begin{itemize}
      \item Can we define upper/lower bounds on the number of hardware types to instantiate?
      \item Should we constrain the topology? 
    \end{itemize}
  \end{itemize}
\end{frame}

\begin{frame}{Solution Approaches }
  Specification
  \begin{itemize} 
     \item Define processing types, e.g., \texttt{proc\_type(dsp). proc\_type(uC).}
     \item Define communication types, e.g., \texttt{comm\_type(router). comm\_type(bus).}
     \item Mapping options assign a task to a processing type, e.g., \texttt{map(m0, t1, dsp). map(m1, t1, uC).}
  \end{itemize}
\end{frame}

\begin{frame}{Solution Approaches (cntd.)}
    Allocation
  \begin{itemize} 
     \item Allocate number of instances
     \item Maximum number of instances per type must be constrained by upper bound
     \item an 'always safe' lower bound is zero
  \end{itemize}

  Binding
  \begin{itemize} 
     \item First, select type binding 
     \item Second, select specific instance for final binding
  \end{itemize}
\end{frame}

\begin{frame}{Solution Approaches (cntd.)}
  \label{slide:bounds}
  \begin{itemize} 
    \item Can we define upper/lower bounds on the number of hardware types to instantiate?
    \begin{itemize}
      \item General upper bound: number of processing element instances at most number of tasks \[\sum(instances)\leq \vert T\vert\]
      \item Specific upper bound: number of instances of a specific type at most number of mappings that contain that type \[\forall type: \sum(instance_{type}) \leq \sum(m)\mid m=(t, type)\]
      \item Lower bound: Only reliably decidable after type-binding
    \end{itemize}
    % \item Should we constrain the topology? 
  \end{itemize}
\end{frame}

\begin{frame}[fragile]{Solution Approaches (cntd.)}
  \lstinputlisting[basicstyle=\tiny,breaklines=true]{alloc.lp}
  \pause
  4 Answer Sets:
  {\tiny
  \[\{\{(t_1,dsp_1),(t_2,dsp_1)\},\{(t_1,dsp_1),(t_2,dsp_2)\},\{(t_1,dsp_2),(t_2,dsp_1)\},\{(t_1,uC_1),(t_2,dsp_1)\}\}\]
  }
\end{frame}

\begin{frame}{Solution Approaches (cntd.)}
  
\end{frame}


\begin{frame}{Relevant Links}
\end{frame}

\begin{frame}{Switching schemes}
\end{frame}

\end{document}