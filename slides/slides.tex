\documentclass[11pt]{beamer}
\usetheme{Warsaw}
\setbeamertemplate{page number in head/foot}[totalframenumber]
\usepackage[utf8]{inputenc}
\usepackage{amsmath}
\usepackage{amsthm}
\usepackage{amsfonts}
\usepackage{amssymb}
\usepackage{stmaryrd}
\usepackage{courier}
\usepackage{xcolor}
\usepackage{relsize}
\usepackage{tikz}
\usepackage{listings}

\newcommand{\Underscore}{\textscale{1}{\textunderscore}}
\author[Müller et al.]{Luise Müller and Kai Neubauer and Philipp Wanko}
\title[DSE with ASP: Progress and Outlook]{DSE with ASP: Progress and Outlook}
%\setbeamercovered{transparent} 
%\setbeamertemplate{navigation symbols}{} 
%\logo{} 
%\institute{} 
\date{} 
%\subject{} 

\begin{document}
\input{listings}
\input{systems}

\begin{frame}
\titlepage
\end{frame}

\section{Overview}

\begin{frame}{Overview}
\begin{itemize}
  \item Consolidated old work
  \item Reimplemented system with \clingo's application class
  \pause
  \item Currently working on experiments and writeup for \emph{Evolutionary System Design}
  \pause
  \item Collecting material and ideas for \emph{Generative Design Space Exploration}
\end{itemize}
\end{frame}

\section{Evolutionary System Design}

\begin{frame}{Problem Description}
  \begin{itemize}
    \item Given is a legacy specification
    \item Specification is modified
    \item Given modified specification and implementation of the legacy specification, find implementations that are, first, Pareto-optimal and, second, close to the legacy implementation
  \end{itemize}
\end{frame}

\begin{frame}{Experimental Setup}
  \begin{enumerate}
    \item Find best-possible implementations to our benchmark set with a high timeout - currently running
    \item Select among the non-dominated implementations a random implementation - TODO
    \item Modify instance benchmark set to varying degrees - TODO
    \item Find implementations to new benchmarks with small timeout - TODO
    \item Employ strategies, optimization, domain-specific heuristics to find implementations to new benchmarks that are similar to legacy solution with same timeout as previous step - TODO
  \end{enumerate}
  \pause
  Expectation: We have faster convergence and smaller distance with similarity information. 
  This gives us high-quality solutions faster while being easier to produce.
\end{frame}

\begin{frame}{Strategies}
	\begin{itemize}
		\item Prevent that unequal design decisions are made
		\item Constrain the search space
		\item E.g. If possible, prevent that the same task is bound on different resources and force this way equal bindings 
		{\small\lstinputlisting[language=clingo,breaklines=true, firstline=1,lastline=1]{listings.lp}}
		\pause
		\item Similarly, force equal Routing and equal Scheduling if possible
	\end{itemize}
\end{frame}

\begin{frame}{Optimization}
	\begin{itemize}
		\item Implementation of a similarity metric based on equal / unequal design decisions
		\item Similarity metric can be used, first, for optimization during exploration and, second, for evaluating how close the resulting implementations are to the legacy implementation
		\newline
		\pause 
		\item TODO - questions: How to implement? (e.g. using Hamming distance), How to weight the system synthesis steps?
	\end{itemize}
\end{frame}

\begin{frame}{Heuristics}
	\begin{itemize}
		\item Prefer equal / unequal design decisions
		\item Direct the search
		\pause
		\item Modular implementation by the use of constants
		{\small\lstinputlisting[language=clingo,breaklines=true, firstline=3,lastline=4]{listings.lp}}
		\pause
		{\small\lstinputlisting[language=clingo,breaklines=true, firstline=5,lastline=5]{listings.lp}}
		\pause
		\item Considered heuristic modifier: true, false, factor, sign, level
	\end{itemize}
\end{frame}

\begin{frame}{Heuristics (cntd.)}
	\begin{itemize}
		\item Different concepts for heuristics
		\item Concept 1
		{\small\lstinputlisting[language=clingo,breaklines=true, firstline=7,lastline=7]{listings.lp}}
		\pause
		\item Concept 2
		{\small\lstinputlisting[language=clingo,breaklines=true, firstline=8,lastline=8]{listings.lp}}
		\pause
		\item As well, for both a version for not equal decisions
		\item Similarly, for Routing and Scheduling decisions
	\end{itemize}
\end{frame}

\section{Generative Design Space Exploration}

\begin{frame}{Problem Description}
  \begin{itemize}
    \item Architecture unknown prior to design process
    \item Instead of structure, we only have a set of hardware types (i.e., \textit{device library})
    \item As before, classified into processing and communication elements
    \item Mapping options assign tasks to processing \textit{types}, not to specific instances
    \item Topology of architecture is subject to exploration 
    \item Research questions:
    \begin{itemize}
      \item Can we define upper/lower bounds on the number of hardware types to instantiate?
      \item Should we constrain the topology? 
    \end{itemize}
  \end{itemize}
\end{frame}

\begin{frame}{Solution Approaches }
  Specification
  \begin{itemize} 
     \item Define processing types, e.g., \texttt{procType(dsp). procType(uC).}
     \item Define communication types, e.g., \texttt{commType(router). commType(bus).}
     \item Mapping options assign a task to a processing type, e.g., \texttt{map(m0, t1, dsp). map(m1, t1, uC).}
  \end{itemize}
\end{frame}

\begin{frame}{Solution Approaches (cntd.)}
    Allocation
  \begin{itemize} 
     \item Allocate number of instances
     \item Maximum number of instances per type must be constrained by upper bound
     \item an 'always safe' lower bound is zero
  \end{itemize}

  Binding
  \begin{itemize} 
     \item First, select type binding 
     \item Second, select specific instance for final binding
  \end{itemize}
\end{frame}

\begin{frame}{Solution Approaches (cntd.)}
  \label{slide:bounds}
  \begin{itemize} 
    \item Can we define upper/lower bounds on the number of hardware types to instantiate?
    \begin{itemize}
      \item General upper bound: number of processing element instances at most number of tasks \[\sum(instances)\leq \vert T\vert\]
      \item Specific upper bound: number of instances of a specific type at most number of mappings that contain that type \[\forall type: \sum(instance_{type}) \leq \sum(m)\mid m=(t, type)\]
      \item Lower bound: Only reliably decidable after type-binding
    \end{itemize}
    % \item Should we constrain the topology? 
  \end{itemize}
\end{frame}

\begin{frame}[fragile]{Solution Approaches (cntd.)}

  {\tiny\lstinputlisting[language=clingo,numbers=left,breaklines=true]{alloc.lp}}
  \pause
  4 Answer Sets: (only allocation \& binding!)
  {\tiny
  \[\{\{(t_1,dsp_1),(t_2,dsp_1)\},\{(t_1,dsp_1),(t_2,dsp_2)\},\{(t_1,dsp_2),(t_2,dsp_1)\},\{(t_1,uC_1),(t_2,dsp_1)\}\}\]
  }
\end{frame}

\begin{frame}{Solution Approaches (cntd.)}
  Topology
  \begin{itemize}
    \item So far, only binding and allocation is decided
    \item Connection between processing elements is not decided yet
    \item Must be conducted through available communication types and links
    \item Theoretically, infinite connection patterns
    \item Practically, upper bound depends on the number of messages in the application
  \end{itemize}
\end{frame}

\begin{frame}{Solution Approaches (cntd.)}
  Topology
  \begin{itemize}
  \item However: How many connections do we allow for each communication device?
    \begin{itemize}
      \item Could be defined in device library: e.g., \texttt{commType(bus;router). allowedConnections(bus,20). allowedConnections(router,5)\footnote{North, East, South, West, Home}.}
      \item Self-edges should not be allowed (probably?!)
    \end{itemize}
    \item Processing instances do not have to be connected if no message is routed between them
    \item Maybe reuse code from instance generator\ldots
  \end{itemize}
\end{frame}

\begin{frame}{Relevant Links}
  \begin{itemize}
    \item Some links can be discarded for the routing
    \item If (r1,r2) is used, (r2,r1) can be discarded (sic!)
    \item DOR: (r1,r2) can be discarded if always leading to the wrong direction
    \item DOR: (r1,r2) can be discarded if one is always in the wrong direction
  \end{itemize}
\end{frame}

\end{document}